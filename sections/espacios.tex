\section{Espacios como n-uplas}

%% Desigualdad de Cauchy-Schwartz %%
\subsection{Desigualdad de Cauchy-Schwartz}
\teorema{\textbf{\underline{Formalización}}: $\lvert<P, Q>\rvert \leq \lvert\lvert P \rvert\rvert * \lvert\lvert Q \rvert\rvert$}

%% Conjunto abierto %%
\subsection{Conjunto abierto}
\teorema{Un conjunto $C$ es abierto si para cada punto $P \in C, \exists \text{ bola abierta } B_r(P) \text{ / } B_r(P) \subset C$}

%% Bola abierta es un conjunto abierto %%
\subsection{Bola abierta es un conjunto abierto}
\teorema{\textbf{\underline{Formalización}}: $\text{Sea } B_r(P) \text{ una bola abierta en } \mathbb{R}^n, \forall \ Q \in B_r(P), \exists t > 0 \text{ / } B_t(Q) \subset B_r(P).$
\vskip 8pt
\noindent \textbf{\underline{Observación}}: En particular, una bola abierta es un conjunto abierto.}

%% Punto interior %%
\subsection{Punto interior}
\teorema{\textbf{\underline{Formalización}}: Un punto $P \in C \subset \mathbb{R}^n$ es interior si $\exists \ r > 0 \text{ / } B_r(P) \subset C.$
\vskip 8pt
\noindent \textbf{\underline{Observación}}: El conjunto de todos los puntos interiores de $C$ se denomina \textbf{interior de C} y se denota $C^0$.}

%% Punto de acumulación %%
\subsection{Punto de acumulación}
\teorema{\textbf{\underline{Formalización}}: Si existe $\{P_n\} \subset C \text{ / } \lim P_n = P \implies P \text{ es un punto de acumulación de } C.$
\vskip 8pt
\noindent \textbf{\underline{Observación}}: El conjunto de todos los puntos de acumulación de $C$ se denomina \textbf{clausura de C} y se denota $\bar{C}$.}

%% Conjunto cerrado %%
\subsection{Conjunto cerrado}
\teorema{\textbf{\underline{Formalización}}: $C \subset \mathbb{R}^n \text{ es cerrado } \iff \big(\ \forall \ \{P_n\} \subset C \implies \lim P_n = P \in C \ \big)$}

%% C cerrado sii  C^c abierto %%
\subsection{$C \subset \mathbb{R}^n \text{ cerrado } \iff C^c \text{ abierto }$}