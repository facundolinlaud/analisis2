\section{Derivadas y diferencial}
%% Gradiente %%
\subsection{Notación: gradiente y plano tangente}
\teorema{
	\begin{equation*}
		\begin{split}
			\text{\textbf{Gradiente de f en p:}} \ & \nabla f_P = (f_{x_1}(P), \dots, f_{x_n}(P)) \\
			\text{\textbf{Diferencial de f en p:}}  \ & Df_P(Y) = \ <\nabla f_P, Y> \ = \sum_{i=1 \dots n} f_{x_i}(P) * y_i \\
			\text{\textbf{Plano tg a f en p:}} \ & x_{n+1} = f(P) \ + <\nabla f_P, X - P> \\
			\text{\textbf{Desigualdad Cauchy–Schartz:}} \ & \modulo{Df_P(X)} = \modulo{<\nabla f_P, X>} \leq \norma{\nabla f_P} \cdot \norma{X}
		\end{split}
	\end{equation*}
}

%% Diferenciabilidad %%
\subsection{Diferenciabilidad}
\teorema{
	\textbf{\underline{Formalización}}:
	\begin{equation*}
		\begin{split}
			f \text{ diferenciable en } p
		\end{split}
		\qquad\Longleftrightarrow\qquad
		\begin{split}
			&f: A \subset \mathbb{R}^n \to \mathbb{R}^m \\
			&P \in A^0 \\
			&\text{Las derivadas parciales de } f \text{ existen en } P \\
			&lim_{X \to P} \frac{\lvert f(X) - f(P) - <\nabla f_P, X - P> \rvert}{\norma{X - P}}=0
		\end{split}
	\end{equation*}
}

%% Diferenciable => Continua %%
\subsection{Si $f$ diferenciable en $P$ $\implies f$ contínua en $P$}
\teorema{
	\textbf{\underline{Formalización}}:
	\begin{equation*}
		\begin{split}
			&P \in A^0 \\
			&f: A \subset \mathbb{R}^n \to \mathbb{R} \\
			&f \ \text{diferenciable en } P
		\end{split}
		\qquad\Longrightarrow\qquad
		\begin{split}
			f \text{ contínua en } P
		\end{split}
	\end{equation*}
	\vskip 8pt
	\textbf{\underline{Idea}}: quiero ver que $\modulo{f(X) - f(P)} \to 0$ con $x \to P$
	\begin{equation*}
		\begin{split}
			\modulo{f(X) - f(P)} \\
			\\ \\ \\
		\end{split}
		\quad
		\begin{split}
		= \\ \\ \\ \\
		\end{split}
		\quad
		\begin{split}
			&\modulo{f(X) - f(P) - Df_P(X-P) + Df_P(X-P)} \\
			&\modulo{f(X) - f(P) - Df_P(X-P)} + \modulo{Df_P(X-P)} \\
			&\modulo{f(X) - f(P) - Df_P(X-P)} \ + <\nabla f_P, X - P> \\
			&\modulo{f(X) - f(P) - Df_P(X-P)} + \norma{\ \nabla f_P \ } \cdot \norma{\ X-P\ }
		\end{split}
		\quad
		\begin{split}
		&\leq \\ &\leq \\ &\leq \text{(c-s)} \\ &\to 0
		\end{split}
	\end{equation*}
	Porque:
	\begin{itemize}
		\item $\modulo{f(X) - f(P) - Df_P(X-P)} = \frac{\modulo{f(X) - f(P) - Df_P(X-P)}}{\norma{X-P}} \cdot \norma{X-P}$
		\begin{itemize}
			\item Como $f$ es diferenciable, la cosa grande tiende a $0$.
			\item $\norma{X-P}$ también tiende a $0$ dado que $X \to P$
		\end{itemize}
		\item $\norma{\nabla f_P}$ es anulado por $\norma{X-P}$
	\end{itemize}
}

%% Si existe transformación lineal diferencial => existen todas las direccionales %%
\subsection{Si existe transformación lineal diferencial $\implies$ existen todas las direccionales}
\teorema{
	\textbf{\underline{Formalización}: si}
	\begin{itemize}
		\item $f: A \subset \mathbb{R}^n \to \mathbb{R}$
		\item $P \in A^0$
		\item $\exists \ T_P: \mathbb{R}^n \to \mathbb{R} \text{ / } \lim_{X \to P} \frac{\modulo{f(X) - f(P) - T_P(X-P)}}{\norma(X-P)}=0$
	\end{itemize}
	\textbf{entonces}:
	\begin{enumerate}
		\item Existen todas las derivadas direccionales de $f$ en $P$ y vale: $f_V(p) = T_P(V) \ \forall \ V \in \mathbb{R}^n$, con $\norma{V} = 1$
		\item En particular, existen todas las parciales (utilizando los vectores canónicos)
		\item Se tiene $T_P(X) = Df_P(X) = <\nabla f_P, X> \forall \ X \in \mathbb{R}^n$ y $f$ diferenciable en $P$
	\end{enumerate}
	\textbf{\underline{Idea}}: quiero ver que $\lim_{t \to 0} \frac{\modulo{f(P + tV) - f(P)}}{t}=0$
	\begin{equation*}
		\begin{split}
			\lim_{X \to P} \frac{\modulo{f(X) - f(P) - T_P(X-P)}}{\norma{X-P}} \quad &= \quad \lim_{X \to P} \frac{\modulo{f(X) - f(P) - T_P(X-P)}}{\norma{X-P}} \\
			\quad &= \quad \lim_{t \to 0} \frac{\modulo{f(P + tV) - f(P) - t \cdot T_P(V)}}{\norma{t}} \\
			\quad &= \quad \lim_{t \to 0} \ \Bigl| \frac{f(P + tV) - f(P) - t \cdot T_P(V)}{t} \Bigr| \\
			\quad &= \quad \lim_{t \to 0} \ \Bigl| \frac{f(P + tV) - f(P)}{t} - T_P(V) \Bigr| \\
			&= 0 \text{, dado que } f \text{ es diferenciable.}
		\end{split}
	\end{equation*}
	\begin{enumerate}
		\item $\lim_{t \to 0} \ \Bigl| \frac{f(P + tV) - f(P)}{t}\Bigr| = T_P(V)$ y $T_P(V)$ está definida para todo $V \in \mathbb{R}^n$ por ser $T$ transformación lineal.
		\item En particular, tomando cualquier $E_i$ se tiene que el límite existe, por lo tanto las parciales existen
		\item Te la debo % TODO
	\end{enumerate}
}

%% Fermat %%
\subsection{Si $f$ derivable en $c \in (a, b)$ y $c$ extremo local $\implies f'(c) = 0$ (Fermat $\mathbb{R}$)}
\teorema{
	\textbf{\underline{Formalización}:}
	\begin{equation*}
		\begin{split}
			&f: [a, b] \to \mathbb{R} \\
			&f \text{ derivable en } (a, b) \\
			&c \in (a, b) \text{ extremo local}
		\end{split}
		\qquad\Longrightarrow\qquad
		\begin{split}
			&f'(c) = 0
		\end{split}
	\end{equation*}
	\textbf{\underline{Idea}}:
	\begin{enumerate}
		\item Supongamos que $f$ tiene un máximo local en $c$, luego existe un $\epsilon > 0 \text{ / } \forall \ x \in (c - \epsilon, c + \epsilon): f(x) \leq f(c)$
		\item Si nos acercamos a $c$ por izquierda: $\lim_{t \to 0^-} \frac{f(c + t) - f(c)}{t} \geq 0$
		\item Y si lo hacemos por derecha, tenemos: $\lim_{t \to 0^+} \frac{f(c + t) - f(c)}{t} \leq 0$
		\item Por lo tanto, la única posiblidad que queda es que $\lim_{t \to 0} \frac{f(c + t) - f(c)}{t} = 0$, o sea $f'(c) = 0$
	\end{enumerate}
	\textbf{\underline{Observación}:} El caso de extremo mínimo es análogo
}

%% Rolle %%
\subsection{Si $f$ contínua en $[a, b]$, derivable en $(a, b)$ y $f(a) = f(b) \implies \text{existe } c \in (a, b) \text{ / } f'(c) = 0$ (Rolle en $\mathbb{R}$)}
\teorema{
	\textbf{\underline{Formalización}:}
	\begin{equation*}
		\begin{split}
			&f: [a, b] \to \mathbb{R} \\
			&f \text{ contínua en } [a, b] \\
			&f \text{ derivable en } (a, b) \\
			&f(a) = f(b)
		\end{split}
		\qquad\Longrightarrow\qquad
		\begin{split}
			&\exists \ c \in (a, b) \text{ / } f'(c) = 0
		\end{split}
	\end{equation*}
	\textbf{\underline{Idea}}:
	\begin{enumerate}
		\item Si $f$ constante entonces $\forall \ x \in (a, b): f'(x) = 0$
		\item Si $f$ no es constante, como es contínua en $[a, b]$ y derivable en $(a, b)$, entonces – por Weierstrass – su imagen está acotada y sus extremos son alcanzados (también se podría decir que su imagen es cerrada por corolario)
		\item Sea $c \in (a, b)$ un extremo en $f$, luego, por Fermat: $f'(c) = 0$
	\end{enumerate}
}

%% Lagrange %%
\subsection{Si $f$ contínua en $[a, b]$ y derivable en $(a, b) \implies \text{existe } c \in (a, b) \text{ / } f'(c) = \frac{f(b) - f(a)}{b-a}$ (Lagrange en $\mathbb{R}$)}
\teorema{
	\textbf{\underline{Formalización}:}
	\begin{equation*}
		\begin{split}
			&f: [a, b] \to \mathbb{R} \\
			&f \text{ contínua en } [a, b] \\
			&f \text{ derivable en } (a, b)
		\end{split}
		\qquad\Longrightarrow\qquad
		\begin{split}
			&\exists \ c \in (a, b) \text{ / } f'(c) = \frac{f(b) - f(a)}{b-a}
		\end{split}
	\end{equation*}
	\textbf{\underline{Idea}}:
	\begin{enumerate}
		\item Sea $L$ la función lineal que conecta las dos puntas de $f$, o sea: $(a, f(a))$ con $(b, f(b))$
		\item Y sea $g(x) = f(x) - L(x)$ que es contínua y derivable pues $f$ lo es y $L$ es función lineal
		\item Como $g(a) = f(a) - L(a) = 0$ y $g(b) = f(b) - L(b) = 0 \implies g(a) = g(b)$
		\item Por Rolle, existe $c \in (a, b) \text{ / } g'(c) = 0$
		\item Tomemos la derivada de $g$ en $c$, que es: $g'(c) = f'(c) - L'(c) = 0$
		\item Pasando para el otro lado: $f'(c) = L'(c)$
		\begin{enumerate}
			\item Recordemos que $L$ es de la forma $mx + k$, por lo tanto $L'(x) = m$ con $m$ pendiente de $L$
		\end{enumerate}
		\item Por lo tanto $f'(c) = m$ donde $m = \frac{\nabla y}{\nabla x} = \frac{f(b) - f(a)}{b-a}$
		\item Finalmente $f'(c) = \frac{f(b) - f(a)}{b-a}$, como se quería probar
	\end{enumerate}
}